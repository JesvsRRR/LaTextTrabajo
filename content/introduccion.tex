
\section*{Introducción}
\addcontentsline{toc}{section}{Introducción}


La organización eficiente de datos es un aspecto fundamental en el campo de la informática y la programación, ya que influye directamente en el rendimiento de las aplicaciones y sistemas. Los algoritmos de ordenamiento son herramientas esenciales que permiten estructurar conjuntos de datos en un orden específico, facilitando así su búsqueda y manipulación. En este sentido, la elección del algoritmo adecuado puede marcar la diferencia en la eficiencia de un programa, especialmente cuando se trabaja con grandes volúmenes de información.

Esta monografía se centra en la implementación y comparación de los siete algoritmos de ordenamiento más utilizados: Bubble Sort, Selection Sort, Insertion Sort, Counting Sort, Heap Sort, Merge Sort y Quick Sort. A través de un análisis detallado de cada uno de estos métodos, se busca no solo entender su funcionamiento, sino también evaluar su desempeño en diferentes escenarios. La importancia de este estudio radica en que, al comprender las características y limitaciones de cada algoritmo, los desarrolladores pueden tomar decisiones más informadas que optimicen el rendimiento de sus aplicaciones.

El objetivo principal de esta investigación es proporcionar una visión clara y concisa sobre los algoritmos de ordenamiento, destacando sus ventajas y desventajas, así como su aplicabilidad en situaciones prácticas. Además, se pretende contribuir al aprendizaje de los estudiantes de ingeniería de sistemas, ofreciendo un recurso que les permita profundizar en el tema y aplicar este conocimiento en sus futuros proyectos. A lo largo de este documento, se presentarán los resultados de las implementaciones realizadas, así como un análisis comparativo de los tiempos de ejecución, lo que permitirá establecer conclusiones sobre la eficiencia de cada algoritmo en función de diferentes variables.









