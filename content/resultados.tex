\cleardoublepage
\section{Conclusión}

La presente monografía ha permitido analizar y comparar la implementación de los siete principales algoritmos de ordenamiento, revelando que, aunque cada uno de ellos tiene su propia utilidad, su eficiencia varía significativamente en función del tamaño y la naturaleza de los datos a ordenar. Algoritmos como Quick Sort y Merge Sort se destacan por su capacidad para manejar listas grandes de manera eficiente, mientras que métodos más simples como Bubble Sort, a pesar de su facilidad de implementación, muestran un rendimiento inferior en términos de tiempo de ejecución. 

La elección del algoritmo más adecuado debe considerar el contexto específico de la aplicación; por ejemplo, en situaciones donde se trabaja con conjuntos de datos pequeños o casi ordenados, algoritmos como Insertion Sort pueden resultar más efectivos. Asimismo, es fundamental que los desarrolladores comprendan la teoría subyacente a cada algoritmo, ya que esto les permitirá tomar decisiones informadas que optimicen el rendimiento de sus aplicaciones. 

Este estudio no solo contribuye al entendimiento de los algoritmos de ordenamiento, sino que también actúa como una herramienta educativa valiosa para los estudiantes de ingeniería de sistemas, facilitando una mejor comprensión de la complejidad algorítmica y su impacto en el rendimiento computacional. Finalmente, se sugiere que futuras investigaciones se enfoquen en la implementación de algoritmos híbridos y en la optimización de los algoritmos existentes, así como en su evaluación en diversos entornos de programación y hardware, lo que podría abrir nuevas vías para mejorar la eficiencia en el procesamiento de datos.